%--- react------------------------------------------------------------------------------
\begin{frame}[fragile] \frametitle{React render functions}
\begin{CodeBox}{}
function HelloWorld() {
  return (<h1>Hello, world!</h1>);
}
\end{CodeBox}

\vspace{8mm}
A render function:
\begin{itemize}
  \item returns a DOM like tree
  \item easy to create instances
  \begin{itemize}
    \item uses JSX to describe the structure
    \item babel translates JSX to JavaScript code - builds a tree of react elements
  \end{itemize}
  \item react injects the tree into the DOM
  \item re-render when needed
\end{itemize}

\end{frame}

%--- JSX ------------------------------------------------------------------------------
\section{JSX - Build the DOM with expressions}
%--- JSX------------------------------------------------------------------------------
\begin{frame}[fragile] \frametitle{JSX}
\begin{itemize}
  \item looks like html, built from ``html tags'' and react components
  \item all tags must be closed, xml syntax works: \code{<img />}
  \item must be one root tag
  \item can use multiple lines
  \item use () to avoid automatic semicolon insertion!!!
  \item Babel compiles JSX down to \code{React.createElement()} calls
\end{itemize}

\begin{CodeBox}{}
const element = (
<span>
  <h1>Hello, world!</h1>
  <p>Some more text...</p>
</span>);
\end{CodeBox}
\end{frame}

%--- JSX attributes------------------------------------------------------------------------------
\begin{frame}[fragile] \frametitle{JSX attributes}
\begin{itemize}
  \item JSX tags can have attributes
  \item React DOM uses camelCase
  \item html: \code{class}, JSX: \code{className}
  \item html: \code{for}, JSX: \code{htmlFor}
  \item mapps to html attributes when injected to the DOM
\end{itemize}

\vspace{8mm}
\begin{CodeBox}{}
const element1 = <div tabIndex="0"></div>;
const element2 = <img src="picture.jpeg" />;
\end{CodeBox}
\end{frame}

%--- Embedded JavaScript ------------------------------------------------------------------------------
\begin{frame}[fragile] \frametitle{Embedded JavaScript}
JSX can contain embedded JavaScript
\begin{itemize}
  \item syntax: \{ \emph{JavaScript expression} \}
  \item use in:
  \begin{itemize}
    \item attribute values
    \item tag content
  \end{itemize}
  \item the embedded JavaScript expression may evaluate to another JSX expression
\end{itemize}
\vspace{3mm}
\begin{CodeBox}{}
const name = 'Per';
const element1 = <h1>Hello {name}</h1>;

const imgSrc = 'picture.jpeg';
const element2 = <img src={imgSrc} />;

const element3 = (<span>{element1}
  <p>You are transformed to {doSomeMagic(name)}</p></span>);
\end{CodeBox}
\end{frame}

%--- Condition and iteration ------------------------------------------------------------------------------
\begin{frame}[fragile] \frametitle{Condition and iteration}
Use JavaScript for conditional rendering and iteration
\begin{CodeBox}{}
function MyWarning(props) {
  return (<h1>  {props.message ? 
             "Warning: " + props.message : "Well done" }
           </h1>);
}

function TodoList({arrayOfTodos}) {
  return (<ul>
              {arrayOfTodos.map(todo => <li>{todo}</li>)};
            </ul>);
}
\end{CodeBox}
\end{frame}

%--- React Components ------------------------------------------------------------------------------
\section{React Component}
%--- React Component------------------------------------------------------------------------------
\begin{frame}[fragile] \frametitle{Rect Component}
react component:
\begin{itemize}
  \item JavaScript function
  \begin{itemize}
    \item called when the components need rendering
    \item must return a react DOM (a JSX expression)
  \end{itemize}
  \item instantiate it in JSX using the function name, example: \code{<HelloWorld />}
  \item function name must start with capital letter
  \item in JSX:
  \begin{itemize}
    \item lowercase tag name - standard html element
    \item uppercase tag name - a react component
  \end{itemize}
\end{itemize}
\end{frame}

%--- Component example------------------------------------------------------------------------------
\begin{frame}[fragile] \frametitle{Component Example}
\begin{CodeBox}{}
  function HelloWorld(){
    return (
      <>
        <h1>Hello</h1>
        <p>EDAF90 - web programming</p>
      </>
    );
  }
\end{CodeBox}
\end{frame}
%--- Attributes------------------------------------------------------------------------------
\begin{frame}[fragile] \frametitle{Attributes - \code{props}}
a parent can pass data to a child
\begin{itemize}
  \item parent set the value using attributes in JSX
  \item the values are passed in a \code{props} object
  \item you can use any JavaScript value, including arrays, objects and functions
\end{itemize}
\end{frame}

%--- Properties Example ------------------------------------------------------------------------------
\begin{frame}[fragile] \frametitle{Prop Example}

\begin{CodeBox}{}
const element = (<Hello name="Per"/>);

function Hello(props) {
  return (
    <span>
      <h1>Hello {props.name}</h1>
    </span>
  );
}
\end{CodeBox}
\end{frame}

%--- Properties children Example ------------------------------------------------------------------------------
\begin{frame}[fragile] \frametitle{Prop children Example}
\code{props} also contain the children of the component, an object or an array.
\begin{CodeBox}{}
const element = 
(<Hello name="Per">
  I am a child <Button />
</Hello>);

function Hello(props) {
  return (
    <span>
      <h1>Hello {props.name}</h1>
      <p>{props.children}</p>
    </span>
  );
}
\end{CodeBox}
\end{frame}
%--- Event Handling ------------------------------------------------------------------------------
\section{Event Handling}
%--- Handling Events------------------------------------------------------------------------------
\begin{frame}[fragile] \frametitle{Handling Events}
\begin{itemize}
  \item event names are camelCase in JSX
  \item must use \code{preventDefault}, returning \code{false} do nothing
  \item pass a function: \code{onClick =\{myCallbackFunction\}}
  \item called with one argument, a synthetic event according to the W3C spec
\end{itemize}
\begin{CodeBox}{}
function ActionLink() {
  function handleClick(e) {
    e.preventDefault();
    console.log('The link was clicked.');
  }
  return (
    <a href="#" onClick={handleClick}>Click me</a>
  );
}
\end{CodeBox}
\end{frame}

%--- State ------------------------------------------------------------------------------
\section{State}
%--- Component State------------------------------------------------------------------------------
\begin{frame}[fragile] \frametitle{Component State}
State is a components memory:
  \begin{itemize}
    \item preserves data between renderings
    \item managed by react
    \item use state hook to get a snapshot of the state
    \item update using a set function
  \end{itemize}
\end{frame}

%--- State Example ------------------------------------------------------------------------------
\begin{frame}[fragile] \frametitle{useState hook}
\begin{CodeBox}{}
function MyButton() {
  const [count, setCount] = useState(0);

  function handleClick() {
    setCount(count + 1);
  }

  return (
    <button onClick={handleClick}>
      Clicked {count} times
    </button>
  );
}
\end{CodeBox}
\end{frame}

%--- Render Cycle ------------------------------------------------------------------------------
\begin{frame}[fragile] \frametitle{Render Cycle}
\begin{enumerate}
  \item render - root element and all children
  \item commit - update the DOM
  \item wait for external event
  \item call all event handlers, queues all state updates
  \item execute the state updates in the queue
  \item the event handler updates the state
  \item \code{if (Object.is(oldState, newState) goto 1 else goto 3}
\end{enumerate}
\vspace{5mm}The render function must be a pure function of state and \code{props}.
\end{frame}

%--- State over time ------------------------------------------------------------------------------
\begin{frame}[fragile] \frametitle{State over time}
\begin{CodeBox}{use state at render time:}
const [cnt, setCnt] = useHook(0);
handleClick(){
  setCnt(cnt+1);
  setCnt(cnt+1);
}
\end{CodeBox}

\begin{CodeBox}{use the latest state:}
const [cnt, setCnt] = useHook(0);
handleClick(){
  setCnt(newCnt => newCnt+1);
  setCnt(newCnt => newCnt+1);
}
\end{CodeBox}
\end{frame}

%--- Object and Array in State ------------------------------------------------------------------------------
\begin{frame}[fragile] \frametitle{Object and Array in State}
\begin{itemize}
  \item state must be immutable
  \item a new value triggers re-render
  \item compares using \code{Object.is}, must be a new object or array to trigger re-render
  \item copy object: \code{newState = \{...oldState\}}
  \item copy array:
  \begin{itemize}
    \item \code{newState = \[...oldState\]}
    \item \code{newState = oldState.map(}\\
                 \code{(e, index) => index===modifyIndex ? newElement : e)}
  \end{itemize}
  \item safe array functions: 
  \begin{itemize} 
    \item \code{concat, filter, slice, map}
    \item or copy first \code{[...arr]}
  \end{itemize}
  \item deep structures, copy all modified objects: \code{ \{outer..., inner: \{ outer.inner, v: newValue \}\}}
\end{itemize}
\end{frame}

%--- Rules for State ------------------------------------------------------------------------------
\begin{frame}[fragile] \frametitle{Rules for State}
\begin{itemize}
  \item in render functions - treat state as read only
  \item only update state in event handlers
  \item event handlers are local functions in the render function
  \begin{itemize}
    \item \code{[state, setState] = useHook()} is in the closure
    \item \code{setState} do not change the local snapshot
    \item re-render $\rightarrow$ new closure and event handler functions
    \item handlers always have the state viewed by the user
  \end{itemize}
  \item the state must be an immutable data structure, copy and modify
  \item setting the state to a new value triggers a re-rendering of the react tree,
           \\the root component and its children
  \item react updates the state after all event handlers finish
\end{itemize}
\end{frame}

%--- Lifting State Up ------------------------------------------------------------------------------
\begin{frame}[fragile] \frametitle{Lifting State Up}
Child $\rightarrow$ Parent communication
\begin{itemize}
  \item prop pass data down in the tree
  \item data can be any JavaScript value, including functions
  \item pass the \code{setState} function:
  \begin{itemize}
    \item children can update the parents state
    \item only use in handlers, never during rendering
  \end{itemize}
\end{itemize}
Sibling $\leftrightarrow$ Sibling communication
\begin{itemize}
  \item lift upp state - store the state in their nearest common ancestor
  \item drill down props - pass the state trough all intermediate components
\end{itemize}
\end{frame}

%--- Lifting State Up Example ------------------------------------------------------------------------------
\begin{frame}[fragile] \frametitle{Lifting State Up}
\begin{CodeBox}{}
function GreatestCommonAncestor() {
  const [cnt, setCnt] = useState(0);
  function handleClick() { setCnt(cnt + 1); }
  return (<><MyView cnt={cnt} /><MyButton cnt={cnt} onClick={handleClick} /></>)
}

function MyView({ cnt }) {
  return (<p>counter is {cnt}</p>);
}
function MyButton({ cnt, onClick: handleClick }) {
  return (<button onClick={handleClick}>you have clicked me {cnt} times</button>);
}
\end{CodeBox}
\end{frame}

%--- Lists and key ------------------------------------------------------------------------------
\section{Lists and key}
%--- Lists and key------------------------------------------------------------------------------
\begin{frame}[fragile] \frametitle{Lists and key}
\begin{itemize}
  \item JavaScript embedded in JSX may return a collection of react elements
  \item updating the DOM is expensive
  \item react only update parts of the DOM that have been changed
  \item you must help react how an array changes:
  \begin{itemize}
    \item each element must have a \code{key} property
    \item unique among siblings
    \item the value must be preserved over time
    \item avoid array index as key (changes when elements are deleted)
  \end{itemize}
  \item using the key react can detect
  \begin{itemize}
    \item elements changed value
    \item elements have been added to the list
    \item elements have been deleted from the list
  \end{itemize}
\end{itemize}
\end{frame}

%--- Key example------------------------------------------------------------------------------
\begin{frame}[fragile] \frametitle{\code{key} example}

\begin{CodeBox}{}
function MyList(props) { return (
  <ul>
    {props.list.map(row => (
      <li key={row.id}>
        {row.text}
      </li>
    ))}
  </ul>);
}
\end{CodeBox}
\end{frame}

%--- Hooks ------------------------------------------------------------------------------
\section{Hooks ---\\Connecting to react internals}
%--- React Basics ------------------------------------------------------------------------------
\begin{frame}[fragile] \frametitle{React Basics}
Render code
\begin{itemize}
  \item computes the react DOM tree from \code{props} and state
  \item must be pure
  \item no control of when or how often it is called (console.log will be messy)
\end{itemize}
\vspace{5mm}
Event handlers
\begin{itemize}
  \item call is triggered by external events like user actions or network trafic
  \item changes steate
\end{itemize}
\end{frame}

%--- Hooks ------------------------------------------------------------------------------
\begin{frame}[fragile] \frametitle{Hooks}
\begin{itemize}
  \item react basics do not cover all needs
  \item hooks into the internals of react
  \item covers more use cases than react basics
  \item named \code{useSomething()}
  \item you have seen \code{useState()} which provides memory to the componets
\end{itemize}
\vspace{5mm}
Rules of hooks:
\begin{itemize}
  \item called from render code directly, or other hooks
  \item must be called in the same order every re-render
  \item only call from top level (not insida conditions or loops)
\end{itemize}
\end{frame}

%--- Effects ------------------------------------------------------------------------------
\begin{frame}[fragile] \frametitle{Effects}
``side effects that are caused by rendering itself, rather than by a particular event''
\begin{itemize}
  \item example: fetch data from a server when the component is viewed
\end{itemize}
\vspace{5mm}
\begin{CodeBox}{}
import { useEffect } from 'react';

function MyComponent() {
  useEffect(() => {
    // Code here will run after *every* render
  });
  return <div />;
}
\end{CodeBox}
\end{frame}

%--- Effects ------------------------------------------------------------------------------
\begin{frame}[fragile] \frametitle{Effects}
\begin{CodeBox}{}
useEffect(() => {
  // initialisation_code
  return 
    () => {/* clean_up_code */}
  }, [list of dependencies]);
\end{CodeBox}
\vspace{5mm}
\begin{itemize}
  \item \code{initialisation_code} is always followed by \code{clean_up_code}
  \item \code{initialisation_code} is run after component is mounted in the DOM, and when a variable in the dependency list is updated
  \item \code{clean_up_code} is run when the component is removed from the DOM and before \code{initialisation_code} is run due to a value change
\end{itemize}
\end{frame}

%--- Effects - Ticker------------------------------------------------------------------------------
\begin{frame}[fragile] \frametitle{Effects in Ticker example}
\begin{CodeBox}{}
import { useEffect, useState } from 'react';

function Ticker({delay}) {
  const [cnt, setCnt] = useState(0);
  
  useEffect(() => {
    const id = setInterval(_=> setCnt(cnt => cnt+1), delay);
    return () => {clearInterval(id)}
  }, [delay]);
  
  return (<p> ticks: {cnt} </p>);
}
\end{CodeBox}
\end{frame}

%--- Effects - Ticker run------------------------------------------------------------------------------
\begin{frame}[fragile] \frametitle{Effects - Running Ticker}
component mounts:
\begin{itemize}
  \item run \code{initialisation\_code} (closure 1)
\end{itemize}
\code{delay} is changed:
\begin{itemize}
  \item run \code{clean\_up\_code} (closure 1)
  \item run \code{initialisation\_code} (closure 2)
\end{itemize}
\code{delay} is changed:
\begin{itemize}
  \item run \code{clean\_up\_code} (closure 2)
  \item run \code{initialisation\_code} (closure 3)
\end{itemize}
\ldots
component is deleted:
\begin{itemize}
  \item run \code{clean\_up\_code} (closure 3)
\end{itemize}
\end{frame}

%--- Effects - Development mode ------------------------------------------------------------------------------
\begin{frame}[fragile] \frametitle{Effects - Development mode}
In development mode:
\begin{itemize}
  \item components are mounted twice (and deleted once)
  \item stress test the clean up code in effects
  \item will mess upp any \code{console.log}
\end{itemize}
\end{frame}

%--- Context ------------------------------------------------------------------------------
\section{Context ---\\broadcasting in a tree}

%--- Context, broadcasting in a tree ------------------------------------------------------------------------------
\begin{frame}[fragile] \frametitle{Context, broadcasting in a tree}
\begin{itemize}
  \item \code{props} are convenient when communication components are few and close
  \item do not scale
  \item context can broadcast a value to all components in a tree
  \item three steps:
  \begin{itemize}
    \item create the context
    \item provide the value for a subtree
    \item us the value in a component
   \end{itemize}
\end{itemize}
\end{frame}

%--- Context, create ------------------------------------------------------------------------------
\begin{frame}[fragile] \frametitle{Context, create}

\begin{CodeBox}{Create the context in a separate file: MyContext.js}
import { createContext } from "react";

export const MyContext = createContext('default value');
\end{CodeBox}
\end{frame}

%--- Context, provide ------------------------------------------------------------------------------
\begin{frame}[fragile] \frametitle{Context, provide a value}
\begin{CodeBox}{Provide a value in a JSX expression}
import { MyContext } from 'MyContext';

function App() {
  return (
    <MyContext.Provider value={'top level'}>
      <ViewMyContext />
      <MyContext.Provider value={'subtree'}>
        <ViewMyContext />
      </MyContext.Provider>
    </MyContext.Provider>
  );
}
\end{CodeBox}
\end{frame}

%--- Context, use ------------------------------------------------------------------------------
\begin{frame}[fragile] \frametitle{Context, use}

\begin{CodeBox}{Use the context in a component}
import { useContext } from 'react';
import { MyContext } from './MyContext.js';

export function ViewMyContext() {
  const myValue = useContext(MyContext);
  return <p>context is: {myValue}</p>
}
\end{CodeBox}
\end{frame}

%--- Reducer ------------------------------------------------------------------------------
\section{Reducers ---\\Separating state logic}

\begin{frame}[fragile] \frametitle{Reducers}
Using handelrs:
\begin{itemize}
  \item \code{nextState = handler(event, currentState)}
  \item mixing UI code with state logic
  \item hard to reuse state logic in different UI components
\end{itemize}
\vspace{5mm}
Reducers
\begin{itemize}
  \item UI code emits actions
  \item \code{nextState = reducer(action, currentState)}
  \item easy to reuse state logic
  \item clearer which operations are allowed on the state
  \item easy to test state logic
\end{itemize}
\end{frame}

%--- Action ------------------------------------------------------------------------------
\begin{frame}[fragile] \frametitle{Action --- a plain JavaScript object}
\begin{itemize}
  \item \code{type} property, which action to execute on the state
  \item contains all information needed to perform the action
\end{itemize}
\vspace{5mm}
\begin{CodeBox}{}
{
  type: "Added",
  task: { id, text, moreData}
}
\end{CodeBox}
\end{frame}

%--- Reducer ------------------------------------------------------------------------------
\begin{frame}[fragile] \frametitle{Reducer --- a plain JavaScript function}
\begin{CodeBox}{}
function tasksReducer(tasks, action) {
  switch (action.type) {
    case 'deleted': {
      return tasks.filter((t) => t.id !== action.id);
    }
    // more actions
    default: {
      throw Error('Unknown action: ' + action.type);
    }
  }
}
\end{CodeBox}
\end{frame}

%--- useReducer ------------------------------------------------------------------------------
\begin{frame}[fragile] \frametitle{useReducer}
\begin{CodeBox}{}
function MyComponent() {
  const [tasks, dispatch] = useReducer(tasksReducer, initialTasks);

  function handleChangeTask(task) {
    dispatch({
      type: 'changed',
      task: task,
    });
  }
  
  return <button onChange={handleChangeTask}>update</button>}
}
\end{CodeBox}
\end{frame}

%--- TODO ------------------------------------------------------------------------------
\section{Redux ---\\ combining context and reducers}
\begin{frame}[fragile] \frametitle{Redux}
\begin{itemize}
  \item first introduced by facbook
  \item soved the never ending message count bug
  \item simplifies component dependencies
  \item component $\rightarrow$ state $\rightarrow$ component
\end{itemize}
\end{frame}

%--- Redux ------------------------------------------------------------------------------
\begin{frame}[fragile] \frametitle{Redux --- TasksContext.js}
\begin{CodeBox}{}
const TasksContext = createContext(null);
const TasksDispatchContext = createContext(null);

export function useTasks() {
  return useContext(TasksContext);
}

export function useTasksDispatch() {
  return useContext(TasksDispatchContext);
}

function tasksReducer(tasks, action) {
  // your code
}
\end{CodeBox}
\end{frame}

%--- Redux ------------------------------------------------------------------------------
\begin{frame}[fragile] \frametitle{Redux --- TasksContext.js}
\begin{CodeBox}{}
export function TasksProvider({ children }) {
  const [tasks, dispatch] = useReducer(
    tasksReducer,
    initialTasks
  );

  return (
    <TasksContext.Provider value={tasks}>
      <TasksDispatchContext.Provider value={dispatch}>
        {children}
      </TasksDispatchContext.Provider>
    </TasksContext.Provider>
  );
}
\end{CodeBox}
\end{frame}

%--- Redux ------------------------------------------------------------------------------
\begin{frame}[fragile] \frametitle{Redux --- App.js}
\begin{CodeBox}{}
function App() {
  return (
    <TasksProvider>
      <MyAppComponents />
    </TasksProvider>
  );
}
\end{CodeBox}
\end{frame}

%--- Redux ------------------------------------------------------------------------------
\begin{frame}[fragile] \frametitle{Redux --- MyComponent.js}
\begin{CodeBox}{}
function MyComponent() {
  const tasks = useTasks();
  const dispatch = useTasksDispatch();

  function handleAdd(text){
        dispatch({
            type: 'added',
            text: text,
          });
        }
  }
  return (
    <MyUI onAdd={handleAdd} data={tasks} />
  );
}
\end{CodeBox}
\end{frame}

%--- Summary ------------------------------------------------------------------------------
\section{Summary ---\\ Rules of React}
%--- Summary ------------------------------------------------------------------------------
\begin{frame}[fragile] \frametitle{Rules of React}
render code
\begin{itemize}
  \item called by react, when needed
  \item must be pure function of \code{props} and state (\code{useState}, \code{useReducer})
  \item no side effects, do not update state
  \item may use hooks
\end{itemize}
\end{frame}
%--- Summary ------------------------------------------------------------------------------
\begin{frame}[fragile] \frametitle{Rules of React}
event handlers and callback passed to \code{useEffect}
\begin{itemize}
  \item called by the browser
  \item update state to trigger re-render
  \item declare inside component function
  \begin{itemize}
    \item operates on data viewed by the user
    \item \code{setState} and \code{dispatch} in the closure
    \item latest state: \code{setState(currentState => computeNewState)}
  \end{itemize}
\end{itemize}
\end{frame}
%--- Summary ------------------------------------------------------------------------------
\begin{frame}[fragile] \frametitle{Rules of React}
state
\begin{itemize}
  \item owned by react
  \item read by render code, get a snapshot
  \begin{itemize}
    \item \code{useState}
    \item \code{useReducer}
  \end{itemize}
  \item update
  \begin{itemize}
    \item \code{setState} in event handlers and effect callbacks
    \item reducers computes the next state
  \end{itemize}
  \item immutable data structure
\end{itemize}
\end{frame}
%--- Summary ------------------------------------------------------------------------------
\begin{frame}[fragile] \frametitle{Rules of React}
hooks
\begin{itemize}
  \item called by your render code and custom hooks
  \item must be called in the same sequence every re-rener
  \item call from top level, not in conditions or loops
  \item may call other hooks
  \item pure functions
\end{itemize}
\end{frame}
%--- Summary ------------------------------------------------------------------------------
\begin{frame}[fragile] \frametitle{Rules of React}
reducer
\begin{itemize}
  \item called by react, once for each a \code{dispath()}
  \item pure function
  \item remember, state is immutable
\end{itemize}
\end{frame}

%--- Bootstrapping, React ------------------------------------------------------------------------------
\begin{frame}[fragile] \frametitle{Bootstrapping React}
\code{ReactDOM.render()}
\begin{itemize}
  \item updates the DOM
  \item takes two parameters:
  \begin{itemize}
    \item a react element (JSX expression)
    \item a DOM element
  \end{itemize}
  \item updates the DOM if the element already was part of the DOM
  \item optimised, only updated the delta
\end{itemize}
\begin{CodeBox}{bootstrap the app}
ReactDOM.render(
   <App />,
  document.getElementById('root')
);
\end{CodeBox}
\end{frame}

%--- TODO ------------------------------------------------------------------------------
\section{TODO, The remaining slides needs updating}

%--- life cycle ------------------------------------------------------------------------------
\begin{frame}[fragile] \frametitle{Component Life Cycle}
the life of a component
\begin{itemize}
  \item mount
  \item update
  \item unmount
\end{itemize}
\end{frame}

%--- Forms------------------------------------------------------------------------------
\begin{frame}[fragile] \frametitle{Forms}
\begin{itemize}
  \item DOM forms contain state
  \item hard for you JavaScript code to read that state
  \item use the \emph{controlled component} pattern
  \begin{itemize}
    \item the react component have the master state
    \item the DOM reflects the component state
    \item event listener (\code{onChange}): update the component state
  \end{itemize}
\end{itemize}
\vspace{5mm}
\begin{CodeBox}{}
<form onSubmit={this.handleSubmit}>
  <input type="text" 
        value={this.state.value}
        onChange={this.handleChange}
  />
</form>
\end{CodeBox}
\end{frame}

%--- Forms, select------------------------------------------------------------------------------
\begin{frame}[fragile] \frametitle{Form, select}
\begin{itemize}
  \item html, uses \code{selected} attribute in an \code{option} tag
  \item react uses a \code{value} attribute in the \code{select} tag
\end{itemize}
\vspace{3mm}
html
\begin{CodeBox}{}
<select>
  <option value="Coconut">Coconut</option>
  <option selected value="lime">Lime</option>
</select>\end{CodeBox}
\vspace{3mm}
react
\begin{CodeBox}{}
<select value={this.state.value} onChange={this.handler}>
  <option value="lime">Lime</option>
  <option value="coconut">Coconut</option>
</select>\end{CodeBox}
\end{frame}

%--- Forms, textarea------------------------------------------------------------------------------
\begin{frame}[fragile] \frametitle{Form, textarea}
\begin{itemize}
  \item html, value is the content
  \item react uses a \code{value} property
\end{itemize}
\vspace{3mm}
html
\begin{CodeBox}{}
<textarea>
  Hello there, some text in a text area
</textarea>
\end{CodeBox}
\vspace{3mm}
react
\begin{CodeBox}{}
const text = 'Hello there, some text in a text area';
const elem = (
  <textarea value={this.state.value}
               onChange={this.handleChange} />;
\end{CodeBox}
\end{frame}

%--- Forms, file input------------------------------------------------------------------------------
\begin{frame}[fragile] \frametitle{Form, file input}
\begin{CodeBox}{}
<input type="file"/>
\end{CodeBox}
\vspace{8mm}

\begin{itemize}
  \item is read-only
  \item must be uncontrolled
\end{itemize}
\end{frame}

%--- Forms call back------------------------------------------------------------------------------
\begin{frame}[fragile] \frametitle{Form, event handlers}
\begin{CodeBox}{}
class NameForm extends React.Component {
  constructor(props) {
    super(props);
    this.state = {value: ''};
    this.handleChange = this.handleChange.bind(this);
    this.handleSubmit = this.handleSubmit.bind(this);
  }

  handleChange(event) {
    this.setState({value: event.target.value});
  }
  handleSubmit(event) {
    alert('A name was submitted: ' + this.state.value);
    event.preventDefault();
  }
\end{CodeBox}
\end{frame}

%--- parametrised event handler------------------------------------------------------------------------------
\begin{frame}[fragile] \frametitle{Parametrised Event Handler}

\begin{CodeBox}{}
<input type="text" name="givenName" 
  value="this.state.givenName">
\end{CodeBox}
\begin{CodeBox}{}
handleInputChange(event) {
  const target = event.target;
  const value =
    target.type === 'checkbox' ?
        target.checked : target.value;
  const name = target.name;

  this.setState({
    [name]: value
  });
}
\end{CodeBox}
\end{frame}

%--- Form refs------------------------------------------------------------------------------
%\begin{frame}[fragile] \frametitle{Form refs}
%\begin{CodeBox}{}
%const input = React.createRef();
%function handleSubmit(event) {
%  alert('A name was submitted: ' + input.current.value);
%  event.preventDefault();
%}
%
%const form = (
%  <form onSubmit={this.handleSubmit}>
%    <label> Name:
%      <input type="text" ref={input} />
%    </label>
%    <input type="submit" value="Submit" />
%  </form>);
%\end{CodeBox}
%\end{frame}



